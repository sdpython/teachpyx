\documentclass[a4paper,11pt]{article}
%\usepackage[utf8]{inputenc}
\usepackage[french]{babel}
\usepackage[T1]{fontenc}
\usepackage{amsmath}
\usepackage{graphicx}
\usepackage{enumitem}
\usepackage[lmargin=2.5cm,rmargin=2.5cm,tmargin=2cm,bmargin=2.5cm]{geometry}
\usepackage{listings}
\usepackage[dvipsnames]{xcolor}

\lstdefinestyle{mypython}{
    language=Python,
    backgroundcolor=\color{white},
    basicstyle=\ttfamily\footnotesize,
    frame=single,
    keywordstyle=\color{blue},
    commentstyle=\color{ForestGreen},
    stringstyle=\color{red},
    numbers=left,
    numberstyle=\tiny\color{gray},
    stepnumber=1,
    tabsize=4,
    showstringspaces=false
}

%\newenvironment{verbatim}{\begin{lstlisting}[style=mypython]}{\end{lstlisting}}
%\newenvironment{verbatim}{\begin{lstlisting}[style=mypython]}{\end{lstlisting}}

\setlength{\parindent}{0pt}

\newcounter{question}
\newcommand{\exequest}[1]{\bigskip \stepcounter{question} \textbf{Question \arabic{question}} : #1}

% Informations sur l'examen
\title{ENSAE TD noté, mercredi 5 novembre 2024}
\date{}

\begin{document}

\vspace{-3cm} 
\maketitle
\vspace{-1.5cm} 

\medskip
\emph{Toutes les questions valent 2 points.}
\bigskip

Au pays des schtroumphs, le schtroumph statisticien a décidé d'organiser le recensement
des schtroumphs. La première tout s'est déroulé à la perfection et a abouti
au résultat suivant :

\\

\begin{tabular}{lrrl}
\toprule
 & schtroumph & schtroumph & index \\
\midrule
0 & 8 & 7 & schtroumph \\
1 & 18 & 18 & schtroumph \\
2 & 6 & 8 & schtroumph \\
\bottomrule
\end{tabular}

\\

L'année suivante, le statisticien a délégué le recensement à son apprenti.

\\

\begin{tabular}{lrrl}
\toprule
 & schtroumph & schtroumph & index \\
\midrule
0 & 18 & 18 & schtroumph \\
1 & 7 & 9 & schtroumph \\
2 & 8 & 6 & schtroumph \\
\bottomrule
\end{tabular}

\\

Au début, il s'est demandé comment la population avait tant changé.
Puis il s'est rappelé que la langue schtroumph est une langue parlée.
Tous les mots s'écrivent de la même façon.
Le schroumph statisticien va devoir réconcilier les données.


%%%%%
\exequest{Implémenter une fonction qui calcule la distance entre deux matrices}

\begin{lstlisting}[style=mypython]
def distance(table1, table2):
    # ...
    return ...

assert distance(np.array([[0, 1], [0, 1]]), np.array([[0, 1], [0, 1]])) == 0
\end{lstlisting}

\exequest{Implémenter une fonction qui retourne les permutations des n premiers entiers}

\exequest{Implémenter une fonction qui permute les colonnes d'une matrice.}

\begin{lstlisting}[style=mypython]
def permute_colonne(table, permutation):
    # ...
    return ...
\end{lstlisting}

\exequest{Faire de même pour les lignes}

\begin{lstlisting}[style=mypython]
def permute_ligne(table, permutation):
    # ...
    return ...
\end{lstlisting}

Faire de même pour les lignes.

\exequest{Ecrire une fonctionne qui retourne les deux permutations ligne/colonne 
qui minimise la distance entre les deux matrices, en déduire la case qui a changé.}

\exequest{Quel est le coût de cette fonction ?}

\exequest{C'est beaucoup trop long. On propose que calculer chaque permutation séparément.}
On cherche donc la meilleure permutation qui minimise la distribution de la somme par ligne
et par colonne entre les deux matrices.
Ecrire une fonctionne qui implémente ce raisonnement.

\exequest{Mais c'est encore trop coûteux}
On cherche la matrice M qui minimise $AM=B$ où A et B sont les sommes sur les colonnes où lignes
des matrices de statistiques observées sur deux années.

\exequest{Comment utiliser cette fonction pour implémenter une version plus rapide de la fonction à la question 5.}

\exequest{La troisième année, une colonne est coupée en deux : une catégorie est divisée en deux sous-catégories.}

Que proposez-vous pour y remédier ?

\end{document}
