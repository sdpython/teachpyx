\input{exo_begin.tex}%

\huge ENSAE TD noté, mardi 8 novembre 2023 \normalsize

Toutes les questions valent 2 points. Il est conseillé de lire l'intégralité de l'énoncé avant de commencer.

\exosubject{}
\begin{xexercice}\label{td_note_label4_2023}

Une langue étrangère s'écrit avec 10 lettres \codesbf{ABCDEFGHUIJ}. Chacune est représentée par 4 bits :

\begin{verbatimx}
A 0000
B 0001
C 0010
D 0011
E 0100
F 0101
G 0110
H 0111
I 1000
J 1001
\end{verbatimx}

Avec cette représentation, \codesbf{00000110} signifie \codesbf{AG}.

%1
\exequest Ecrire une fonction qui code une séquence de lettres en une séquence de 0 et 1.

\begin{verbatimx}
def code(text):
    # ....
    return ...

assert code("AG") == "00000110"
\end{verbatimx}

%2
\exequest On s'intéresse au décodage d'un message. Première étage : écrire une fonction qui retourne la première lettre
correpondant au premier code qui commence un message codé.

\begin{verbatimx}
def first_letter(chaine):
   # ....

assert first_letter("10010001") == "J"
\end{verbatimx}


%3
\exequest Ecrire une fonction qui reçoit une séquence de 0 et de 1 et retourne la séquence de lettres correspondante.

\begin{verbatimx}
def decode(chaine):
    # ....
    return ...

assert decode("00000110") == "AG"
\end{verbatimx}

%4
\exequest On forme une classe avec les fonctions précédentes. Il faut compléter le code suivant.

\begin{verbatimx}
class Coding:
    def __init__(self):
        self.mapping = {'A': '0000', 'B': '0001', 'C': '0010', 'D': '0011',
                        'E': '0100', 'F': '0101', 'G': '0110', 'H': '0111',
                        'I': '1000', 'J': '1001' }

    def first_letter(chaine):
        # ....

    def code(self, text):
        # ...

    def decode(self, chaine):
        # ...

cl = Coding()
assert cl.code("AG") == "00000110"
assert cl.first_letter("00000110") == "A"
assert cl.decode("00000110") == "AG"
\end{verbatimx}

%5
\exequest On veut réduire la taille du message codé.
Les lettres de A à G sont maintenant codées sur 3 bits et les suivantes sur 5.

\begin{verbatimx}
A 000
B 001
C 010
D 011
E 100
F 101
G 110
H 11100
I 11101
J 11110
\end{verbatimx}

On crée une nouvelle classe \codesbf{Coding35} qui hérite de la classe \codesbf{Coding}.

\begin{verbatimx}
class Coding35(Coding):
    def __init__(self):
        # ....

cl = Coding35()
assert cl.code("AH") == "00011100"
assert cl.decode("00011100") == "AH"
\end{verbatimx}

%6
\exequest Que fait la fonction suivante ? Que suppose-t-elle sur la méthode \codesbf{decode} pour qu'elle fonctionne.
\textit{Il n'est pas demandé de modifier votre code pour qu'elle fonctionne.}

\begin{verbatimx}
def which_coding(text, codings):
    return [c for c in codings if c.decode(text) is not None]

codings = [Coding(), Coding35()]
assert which_coding("0000", codings) == codings[:1]
\end{verbatimx}

%7
\exequest Dans ce langage, les lettres sont toutes équiprobables.
Quel code est le plus court pour un texte aléatoire très grand et quantifier le gain ?
Que se passe-t-il si la lettre J a une probabilité de 0.5 et toutes les autres lettres ont la même probabilité
d'apparition ? Que suggérez-vous pour optimiser le Coding en terme de longueur ?
*Aucun code n'est demandé.*

%8
\exequest On change le Coding des lettres A et B : \codesbf{A \, 00} et \codesbf{B \, 01}. Il faut créer une troisième classe
héritant de la première. Que valent \codesbf{c.code("BGBB")} et \codesbf{c.code("DEF")} ?
Que retourne votre méthode \codesbf{decode} ?

\begin{verbatimx}
class Coding235(Coding):
    def __init__(self):
        # ....

c = Coding235()
assert c.code("BGBB") == "011100101"
assert c.code("DEF") == ...
\end{verbatimx}

%9
\exequest Dans le cas précédent, la première lettre peut être soit \textbf{B} soit \textbf{D}.
Ecrire une méthode qui retourne toutes les options pour la première lettre d'un message codé.

\begin{verbatimx}
class Coding235(Coding):
    # ....
    def first_letters(self, chaine):
         # ...
        
c = Coding235()
assert c.first_letters("011100101") == {"B", "D"}
\end{verbatimx}

%10
\exequest Ecrire une méthode \codesbf{decode} qui retourne toutes les solutions par récurrence.

\begin{verbatimx}
class Coding235(Coding):
    # ....
    def decode(self, chaine):
         # ...

c = Coding235()
assert c.decode("011100101") == {"BGBB", "DEF"}
\end{verbatimx}

\end{xexercice}




\input{exo_end.tex}%
