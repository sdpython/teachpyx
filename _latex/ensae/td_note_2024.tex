\documentclass[a4paper,11pt]{article}
\usepackage[french]{babel}
\usepackage[T1]{fontenc}
\usepackage{amsmath}
\usepackage{graphicx}
\usepackage{enumitem}
\usepackage[lmargin=2.5cm,rmargin=2.5cm,tmargin=2.5cm,bmargin=2.5cm]{geometry}
\usepackage{listings}
\usepackage{xcolor} % Optionnel pour la coloration syntaxique

\lstset{
    language=Python,          % Langage du code
    basicstyle=\ttfamily\small, % Taille et police du texte
    keywordstyle=\color{blue},  % Couleur des mots-clés
    commentstyle=\color{green}, % Couleur des commentaires
    stringstyle=\color{red},    % Couleur des chaînes de caractères
    breaklines=true,            % Retour à la ligne automatique
    numbers=left,               % Numérotation des lignes à gauche
    numberstyle=\tiny\color{gray}, % Style de la numérotation
    frame=single,               % Encadrement du code
    captionpos=b,               % Position de la légende
    showspaces=false,           % Pas d'affichage des espaces
    showstringspaces=false      % Pas d'affichage des espaces dans les chaînes
}

\newenvironment{codepy}{\begin{lstlisting}[language=Python]}{\end{lstlisting}}


\newcounter{question}
\newcommand{\exequest}[1]{\textbf{Q\usecounter{question} \stepcounter{moncompteur}} #1}

% Informations sur l'examen
\title{ENSAE TD noté, mardi 6 novembre 2024}

\begin{document}

\maketitle

\textit{Toutes les questions valent 2 points.}

\includegraphics[width=0.5\textwidth]{nom_image.jpg}

25 maisons sont positionnées aux 25 intersections du quadrillage ci-dessus.
Le courant (rouge) arrive à un angle du carré (point A).
Il faut relier chaque maison au courant pour un coût minimal.
Pour cela il faut tirer un câble entre chaque intersection au point A.
Les câbles ne peuvent passer que par les routes (les lignes du quadrillage), pas de diagonales donc.

%%%%%
\exequest{Implémenter une fonction qui calcule la distance L1.}

La distance L1 est définie par $d(x_1,y_1,x_2,y_2) = |x_1 - x_2| + |y_1 - y_2|$.

\begin{codepy}
def distance(x1, y1, x2, y2):
    # ...
    return ...

assert distance(0, 0, 3, 4) == 7
\end{codepy}

%%%%%
\exequest{Calculer la longueur de câble pour relier les 25 maisons.}

\begin{codepy}
def longueur_cable(n=5):
    # ...

assert longueur_cable(5) == 100
\end{codepy}

%%%%%
\exequest{Adapter la fonction pour un rectangle 8x9.}

\begin{codepy}
assert longueur_cable(8, 9) == 540
\end{codepy}

%%%%%
\exequest{Avec deux câbles...}

On dispose de deux câbles : 

\begin{enumerate}
\item un câble ne pouvant relier qu'une maison avec un coût $c_1$ par mètre
\item un câble ne pouvant relier qu'une ou deux maisons avec un coût $c_2$ par mètre
\end{enumerate}

Par conséquent, on peut relier une maison \textit{M1} avec un câble $c_2$ puis relier 
\textit{M1} à une autre maison \textit{M2} avec un câble $c_1$.
On veut savoir quand utiliser tel ou tel câble pour minimiser les coûts.

Ecrire une fonction qui retourne le coût du câblage décrit ci-dessus.

\begin{codepy}
def cout_cablage(x1,y1, x2,y2, c1, c2):
    # ...

assert cout_cablage(1,2, 2,4, 1, 1.5) == 7.5
\end{codepy}

%%%%%
\exequest{Que fait le code suivant et que montre-t-il ?}

\begin{codepy}
    def position_m1(n, c1, c2):
    x = []
    y = []
    for i in range(2*n):
        x.append(i)
        c = cout_cablage(0,i, 0,n, c1, c2)
        y.append(c)
    return x, y

x, y = position_m1(5, 1, 1.5)
print(x)  # [0, 1, 2, 3, 4, 5, 6, 7, 8, 9]
print(y)  # [5.0, 5.5, 6.0, 6.5, 7.0, 7.5, 10.0, 12.5, 15.0, 17.5]
\end{codepy}

%%%%%
\exequest{Idée d'algorithme}

\begin{enumerate}
\item On prend une maison non câblée M1 la plus proche de A.
\item On prend ensuite une maison non câblée M2 la plus proche du coin opposé.
\item On continue jusque à la fin.
\end{enumerate}

On crée une matrice *M*, $n \times n$, initilialisé à -1.
La maison 0,0 est relié avec le câble 0: $M[i,j] =$ le numéro du câble qui la relie.

Ecrire une fonction qui initialise la matrice.

\begin{codepy}
def init(n=5):
    # ...
\end{codepy}

%%%%%
\exequest{Ecrire une fonction qui retourne la maison la plus proche d'une position $(i,j)$ et non câblée.}

\begin{codepy}
def maison_proche(M, x, y):
    # ...

M = init()
assert maison_proche(M, 0,0) == (0, 1)
M[0, 1] = 1
M[1, 0] = 2
M[1, 1] = 3
assert maison_proche(M,0,1) == (0, 2)
\end{codepy}


%%%%%
\exequest{On veut tirer un nouveau câble comme suit :}

Ecrire une fonction qui tire un nouveau câble :

\begin{enumerate}
\item on prend une maison proche de A
\item on met à jour la matrice M
\item on prend une maison proche du coin opposé
\item on met à jour la matrice M
\item une retourne le coût
\end{enumerate}

\begin{codepy}
def nouveau_cable(M, c1, c2):
   # ...

M = init()
g = nouveau_cable(M, 1, 1.5)
assert g == ((0, 1), (4, 4), 8.5)
\end{codepy}


%%%%%
\exequest{Terminer l'algorithme.}

Il suffit d'une boucle pour câbler toutes les maisons. La trouverez-vous ?

\begin{codepy}
def algorithme_cablage(n, c1, c2):
    M = init(n)
    n_cables = n * n // 2
    cables = []
    # for ...
    # ...
    return cables

print(algorithme_cablage(5, 1, 1.5))
\end{codepy}


%%%%%
\exequest{L'algorithme a un défaut, trouverez-vous lequel ?}

\begin{codepy}
\end{codepy}


\end{document}
